\section{Анализ подходов к созданию автономных робототехнических систем} \label{literature}

Для синтеза алгоритмов управления автономными роботами наиболее актуальными являются следующие подходы:
\begin{itemize}
	\item эволюционные алгоритмы;
	\item подход на основе нечеткой логики;
	\item формальный подход;
	\item обучение с подкреплением.

\end{itemize}

Далее упомянутые подходы будут рассмотрены подробнее с целью выявить наилучший для решения задачи адаптивного управления шагающим роботом. 
%Далее упомянутые подходы будут рассмотрены подробнее с целью выявить наилучший для решения задачи управления автономным мобильным роботом. 
Выбор будет происходить по следующим критериям:
\begin{itemize}
	\item сложность формализации;
	\item возможность работать в неизвестной среде;
	\item сложность имплементирования;
	\item требовательность к вычислительным ресурсам.
\end{itemize}

Простота формализации является главным критерием, так как главной задачей является упрощение существующих методов создания алгоритмов управления. 
Работа в заранее неизвестной среде обусловлена адаптивностью алгоритма. 
Критерий сложности имплементирования определяет степень сложности воспроизведения алгоритма для новых начальных условий и конфигураций робота. 
Критерий требовательности к вычислительным ресурсам напрямую влияет на стоимость применения выбранного подхода.


\subsection{Эволюционные алгоритмы}

Эволюционные алгоритмы (ЭА) – это алгоритмы, созданные в качестве методов решения оптимизационных задач и основанные на принципах естественного отбора. 
ЭА моделируют базовые положения в теории биологической эволюции -- процессы отбора (селекции), мутации и воспроизводства. 
Множество агентов, называемое популяцией, эволюционирует согласно правилам отбора в соответствии с целевой функцией. Таким образом, каждому агенту (индивидууму) популяции назначается значение его приспособленности (значение фитнесс-функции, являющейся частным случаем целевой функции) в окружающей среде. 
Размножение и мутация позволяют изменяться агентам и приспособляться к среде \cite{evolution}.

В робототехнике применение ЭА позволяет выполнять различные практические задачи, в частности задачи обследования местности и маневрирования между препятствиями. 
Для агентов (роботов) описывается алгоритм, позволяющий им обмениваться генетической информацией с целью адаптации к решению конкретной задачи. 
В качестве генетической информации выступают различные стратегии управления приводами и системами робота на основе конкретных входных значений с датчиков. 
Такие стратегии могут задаваться изначально как случайно, так и согласно какой-либо эвристике. 
Цель – научить робота взаимодействовать с неизвестной средой. 

Преимущества эволюционного подхода:
\begin{itemize}
	\item возможность использования в задачах, сложно или полностью не поддающихся анализу и формализации;
	\item устойчивость в задачах с изменяющейся и/или частично неизвестной средой;
	\item существование большого количества как естественных, так и созданных человеком вариаций генетических операций.
\end{itemize}

Недостатки эволюционного подхода:
\begin{itemize}
	\item оценка функции приспособленности (фитнесс-функции) для сложных проблем (задачи высокой размерности) требует больших вычислительных мощностей, что часто является фактором, ограничивающим использование алгоритмов искусственной эволюции;
	\item кодирование генетической информации (иногда называется геномом или хромосомами) представляет собой также сложный процесс, пусть и менее сложный, чем формализация всей задачи;
	\item симуляция на настоящих роботах требует большого объёма ресурсов, поэтому обычно синтез алгоритма производят в компьютерных симуляциях, а затем полученные алгоритмы встраиваются в настоящие роботы. Однако процесс создания самой симуляции также является трудоёмким.
\end{itemize}


\subsection{Алгоритмы на основе нечёткой логики}

Нечёткая логика -- это раздел математики, являющийся обобщением логики и теории множеств, главным объектом исследования является нечеткое множество. 
В 1965 году Лотфи Заде расширил классическое понятие множества, допустив, что функция принадлежности принимает не только значения 0 и 1, а интервал от 0 до 1 \cite{fuzzy}. 
Традиционный логический блок компьютера может давать определенный ответ -- правда или ложь, что эквивалентно человеческим да или нет. 
Подход нечеткой логики позволяет давать более гибкие ответы, определяя аналоги человеческим: «возможно да», «не могу сказать», «возможно нет» и так далее. 

Архитектура системы нечеткой логики состоит из четырёх основных частей:
\begin{itemize}
	\item фузификатор (от английского fuzzy -- нечеткий); 
	\item база знаний;
	\item блок обработки; 
	\item дефузификатор.
\end{itemize}

Фузификатор -- модуль трансформирующий входной сигнал в элемент нечеткого множества. 
Входом может быть сигнал измеренный сенсорами например температура или скорость. 
Обычно трансформированный сигнал делится на пять диапазонов:
\begin{itemize}
	\item LP -- вход большой положительный;
	\item MP -- вход средний положительный;
	\item S -- вход малый;
	\item MN -- вход средний отрицательный;
	\item LN -- вход большой отрицательный.
\end{itemize}

Функция, которая  определяет отображение из пространства входов в интервал от 0 до 1, называется функцией принадлежности. 
В основном они используют 3 типа сигналов:
\begin{itemize}
	\item испульсный;
	\item гауссовый;
	\item треугольный или трапецеидальный.
\end{itemize}

Пример треугольного сигнала для перевода напряжения в интервал от 0 до 1 приведен на рисунке \ref{fuzzy}.
\addimghere{fuzzy}{1}{Пример треугольного сигнала}{fuzzy} 

База знаний  -- база созданная экспертами в той области, которой реализуется система. 
В ней содержится набор условных правил.

Блок обработки -- модуль обрабатывающий входное значения, используя правила из базы знаний.

Дефузификатор -- трансформирует обработанный вход из нечеткого логики в классическую.

На рисунке \ref{fuzzysch} показана схема работы данной системы.
\addimghere{fuzzysch}{0.5}{Схема работы системы, использующей нечеткую логику.}{fuzzysch} 

Нечеткую логику используют в робототехнике для создания систем управления, систем стабилизации, экспертных систем и в случаях когда необходимо сделать систему реагирующую как человек или для работы совместно с человеком \cite{fuzzy1}, \cite{fuzzy2}, \cite{fuzzy3}.

Достоинства подхода, использующего нечеткую логику:
\begin{itemize}
	\item подход устойчив к шуму входного сигнала;
	\item архитектура подхода понятна и проста в использовании;
	\item гибкость внесения изменений, достаточно изменить правило в базе знаний.
\end{itemize}

Недостатки подхода, использующего нечеткую логику:
\begin{itemize}
	\item нет систематического подхода для построения архитектуры под каждую задачу;
	\item подход остается понятным, только когда архитектура не сложна;
	\item подход не дает высокой точности.
\end{itemize}


\subsection{Формальный подход}
Формализация – представление в виде формальной системы какой-либо содержательной информации, представленной на естественном языке, путём изъятия логических свойств элементов естественного языка, существенных отношений между этими элементами, а также определение принципов  логической дедукции и критериев различия правильных способов рассуждения от неправильных.

%Исправь себе на адаптивный (только, желательно, в копии этого файла. Так же, как и с введением)
Задача синтеза алгоритма управления автономным роботом также требует формализации задачи, если она решается не методами искусственного интеллекта (ИИ). 
В задаче управления автономным мобильным роботом формализация заключается в составлении перечня всевозможных ситуаций, в которых может оказаться робот, и описание действий (и их порядка), которые необходимо в таких ситуациях предпринимать. 

Преимущества:
\begin{itemize}
	\item полная интерпретируемость алгоритма, т.е. возможность объяснить действия робота в любой ситуации;
	\item возможность оптимизации алгоритма управления.
\end{itemize}

Недостатки:
\begin{itemize}
	\item требование значительных человеческих и вычислительных ресурсов для анализа и формализации задачи, 
а также для составления и оптимизации алгоритма (зачастую уникального, т.е. для которого сложно найти примеры имплементации).
\end{itemize}


\subsection{Обучение с подкреплением}

Обучение с подкреплением -- это класс методов машинного обучения, при котором происходит обучение агента (робота), который не имеет сведений о среде, но имеет возможность производить какие-либо действия в ней. 
Действия агента переводят среду в новое состояние и агент получает от среды некоторое вознаграждение или наказание в соответствии с функцией подкрепления. 
Функция подкрепления определяет цель в процессе обучения с подкреплением и является по сути соответствием между состояниями среды и числом, подкреплением, показывающим желательность, ценность состояния. В качестве функции подкрепления может быть, например, расстояние от робота до определённой точки пространства; высота верхней точки шагающего робота (косвенно сообщающая о том, что робот не упал); уровень заряда батареи бортового аккумулятора.
В данном классе методов большое внимание уделяется поощрению/наказанию не только текущих действий, которые непосредственно привели к положительному/отрицательному подкреплению, но и тех действий, которые предшествовали текущим.
Поэтому в качестве целевой функции выступает сумма подкреплений на определённом промежутке пространства, например, суммарное подкрепление за прохождение определённой траектории.
В то время как функция подкрепления определяет прямую, характерную желательность состояния среды, целевая функция обнаруживает долгосрочную желательность состояний после принятия во внимание состояний, которые последуют за текущим, и подкреплений, соответствующих этим состояниям. 
Например, состояние может повлечь низкое непосредственное подкрепление, но при этом сильно положительно повлиять на суммарную оценку, потому как за ним регулярно следуют другие состояние, которые приносят высокие подкрепления. 
Единственная цель агента состоит в максимизации итогового подкрепления (целевой функции), которое тот получает в процессе длительной работы \cite{reinf_lern}.  

В робототехнике обучение с подкреплением активно применяется в задаче передвижения мобильного робота по лабиринту \cite{reinf_lern1} и в задаче управления манипулятором \cite{reinf_lern2}. 
Также существуют попытки обучать агента выполнять задачи разного плана при помощи одного алгоритма\cite{reinf_lern3}.

Преимущества обучения с подкреплением:
\begin{itemize}
	\item обучившаяся нейросеть позволяет решать общую задачу;
	\item для обучения нейросети требуется меньше вычислительных мощностей, чем в эволюционных методах;
	\item достаточно одного агента для обучения.	
\end{itemize}

Недостатки обучения с подкреплением:
\begin{itemize}
	\item требования к вычислительным ресурсам в случаях многоразмерных задач;
	\item в процессе обучения не всегда находится оптимальная стратегия для получения максимальной долгосрочной награды.
\end{itemize}


\subsection{Вывод по главе}

В данной главе были рассмотрены наиболее актуальные подходы для решения задачи синтеза алгоритма управления -- подход на основе нечеткой логики, эволюционный подход, подход на основе обучения с подкреплением, экспертный подход. 
Рассмотренные алгоритмы были сравнены между собой по критериям -- степень формализации, обобщающая способность, трудозатраты и требовательность к вычислительным ресурсам. 
Результаты сравнения представлены в \mbox{таблице \ref{table:ApDifference}.}

\begin{table}[H]
	\caption{Сравнение подходов к созданию систем управления}\label{table:ApDifference}
	\begin{tabular}{|m{3,8cm}|m{1,9cm}|m{2,7cm}|m{3,2cm}|m{3,2cm}|}
		\hline Критерий & Нечеткая логика & Формальный подход & Эволюционные алгоритмы &  Обучение с подкреплением \\
		\hline Сложность формализации  & Высокая & Высокая & Средняя & Малая  \\
		\hline Обобщающая способность & Нет & Нет & Да & Да \\
		\hline Трудозатраты  & Средне & Много & Средне & Мало \\
		\hline Требовательность к вычислительным ресурсам  & Мало & Мало & Много & Средне \\
		\hline 
	\end{tabular}
\end{table}

Степень формализации -- основной критерий, он показывает количество начальной информации необходимо знать об агенте и как полно необходимо описать кинематику и динамику агента. 
В методах эволюционном подходе и обучении с подкреплением происходит абстрагирование от физических характеристик агентов, благодаря этому трудозатраты на создание алгоритмов управления для сложных систем сокращаются. 
Также, из-за того, что в этих методах не используется описание конкретной задачи, а только функция содержащая общие аспекты работы алгоритма, то и сам алгоритм имеет большую обобщающую способность.

Для работы эволюционных методов необходима симуляция множества агентов, что сильно увеличивает требования к вычислительным ресурсам. 
Описание целевой функции в эволюционных методах и в обучении с подкреплением - относительно простая задача. Однако кодирование генома агента в эволюционных алгоритмах - несравненно больший труд.

В связи с вышесказанным, лучшим из рассмотренных методов по выбранным критериям является методы, использующие обучение с подкреплением.

\clearpage

