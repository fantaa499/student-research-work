\anonsection{Введение}

Особый интерес к шагающим машинам появился в 1950-е годы после окончания второй мировой войны. Первой моделью «стопоходящей» машины была создана П.Л. Чебышевым. 
Способ передвижения с помощью ног, является наиболее распространенным в живой природе. 
Однако в технике он еще не получил заметного применения, прежде всего из-за сложности управления.
Основными преимуществами шагающих роботов -- является большая проходимость на пересеченной местности вплоть до возможности передвигаться прыжками и лазать по наклонным поверхностям по сравнению с колесными роботами \cite{urevich}. 

Применение шагающих роботов актуально при работе в опасных зонах, в том числе при устранении аварий техногенного характера и стихийных бедствий. 
В данных ситуациях необходимо исключить риск причинения вреда спасателям, а проходимости колесных роботов будет недостаточно.
Главными проблемами алгоритмов управления шагающими роботами являются:
\begin{itemize}
	\item сложность разработки алгоритма из-за большего числа решаемых подзадач;
	\item неустойчивость разработанного алгоритма к новым внешним условиям;
\end{itemize}

В данной работе будут рассмотрены основные подходы к созданию алгоритмов управления шагающими роботами, а также их сравнение. 
Цель сравнения выбрать подход для благодаря которому будут решены указанные проблемы. 
Для выбранного подхода будет проведен анализ и сравнения существующих решений и выбор наилучших.

\clearpage
