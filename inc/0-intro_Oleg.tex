\anonsection{Введение}


Несмотря на огромные достижения в области автоматизации, до сих пор существуют сферы человеческой деятельности, автоматизация которых либо крайне неэффективна, либо вообще не представляется возможной, когда, казалось бы, весьма несложные операции (с которыми легко справляется любой малоквалифицированный персонал) вообще не поддаются автоматизации. Заполнить многие из таких «свободных от автоматизации ниш» могли бы высокоадаптивные и автономные мобильные роботы (МР). 


«Интеллектуализация» МР является важнейшим направлением развития МР. Под этим термином здесь понимается повышение уровня адаптивности МР к сложным быстроизменяющимся внешним условиям или повышение степени независимости (автономности) процесса функционирования МР от человека-оператора. С увеличением степени автономности МР упрощается процесс управления роботом, уменьшается отрицательное влияние человеческого фактора, возрастает общая эффективность от применения МР.


Основные трудности при этом состоят в создании алгоритмического обеспечения, позволяющего автоматически управлять движением роботов, используя информацию о его положении относительно инерциальной системы координат и препятствий местности.


Однако при наличии большого числа публикаций, в которых используются как классические, так и современные подходы к синтезу алгоритмов обработки информации и управления, задача остается не решенной в полной мере. Это связано с излишней идеализированностью ее постановки, не учитывающей принципиальную невозможность знания точной математической модели робота, из-за отсутствия учета возможностей и характеристик реальных датчиков и желания авторов решить плохо формализуемую проблему управления при наличии большого количества разнородной информации на основе одного, порой достаточно сложного алгоритма.


Учитывая успехи алгоритмов глубокого обучения в различных плохо формализуемых задачах, было решено проанализировать и сравнить качество синтеза алгоритмов управления автономными роботами  на основе нейронных сетей с классическими не нейросетевыми алгоритмами.


Таким образом, актуальность задачи создания алгоритмов системы управления мобильных роботов на основе нейронных сетей, с одной стороны, определяется востребованностью автономных мобильных роботов, с другой - отсутствием эффективных алгоритмов, способных автономно решать плохо формализуемую задачу управления автономным роботом в различных средах.


\clearpage
