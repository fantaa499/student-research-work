\section{Математическое описание выбранных алгоритмов планирования траекторий движения БЛА в среде с препятствиями}
Все представленные алгоритмы имеют одинаковые входные данные --- это координаты начальной и конечной точек.
Далее будут представлены математические описания выбранных алгоритмов планирования траекторий движения БЛА в среде с препятствиями.

\subsection{Алгоритм Stefan Hrabar}
Получив координату конечной точки B, БЛА начинает полет по вектору $R$ соединяющему точку текущего положения --- $A$ и $B$. 
Координаты этих точек соответственно $X_a$, $Y_a$, $Z_a$ и $X_b$, $Y_b$, $Z_b$. 
Координаты вектора $R$ вычисляются следующим способом:
$$\overrightarrow{R}=B-A=(X_a-X_b,Y_a-Y_b,Z_a-Z_b)$$

По мере поступления новых данных с 3Д-камеры, строится 3Д-карта занятости.
Во время полета карта занятости проверяется на наличие препятствий в пределах цилиндрического объема, в котором безопасно может пролететь мультикоптер.
Цилиндрический объем имеет радиус $R_{sv}$ и длину $L_{sv}$.
Данные параметры выбираются в зависимости от величины БЛА и расстояния необходимого для совершения маневра. 
Также на выбор этих параметров влияют внешние возмущения и ошибки модели и управления.
Конкретные значения находятся опытным путем.

Если препятствие попало в цилиндрический объем, происходит поиска точки облета.
Алгоритм облета показан на рисунке \ref{elipse} и выполняется следующим образом.
Поиск точки облета выполняется по точкам на эллипсе, с центром находящимся в центре видимой части препятствия и c плоскостью перпендикулярной вектору $\vec{R}$.
После прохождения промежуточной точки, БЛА продолжает полет к конечной точке.
Для нахождения геометрического центра препятствия воспользуемся следующими формулами:
$$
{\vec  X}_{c}={\frac  {\sum \limits _{i}{X}_{i}}{N}}, 
{\vec  Y}_{c}={\frac  {\sum \limits _{i}{Y}_{i}}{N}}, 
{\vec  Z}_{c}={\frac  {\sum \limits _{i}{Z}_{i}}{N}}, 
$$

Где, $X_c$, $Y_c$, $Z_c$ --- координаты геометрического центра;

$X_i$, $Y_i$, $Z_i$ --- координаты точек препятствия, измеренные 3Д-камерой и попавшие в безопасный объем;

N --- количество точек препятствий , измеренные 3Д-камерой и попавшие в безопасный объем.

Плоскость проходящая через геометрический центр препятствия и перпендикулярная вектору $\vec{R}$ описывается следующим уравнением:
$$
D(X-X_с)+E(Y-Y_с)+F(Z-Z_с)=0
$$
Где, D, E, F --- координаты вектора $\vec{R}$

Для построения эллипса в этой плоскости решаем следующую систему уравнений, состоящую из канонического уравнения эллипса и уравнения плоскости полученного в предыдущем шаге.
$$
{\begin{cases} 
	D(X - X_c)+E(Y-Y_c)+F(Z-Z_c)=0\\
	\frac{(X-X_c)^{2}}{a}+\frac{(Y-Y_c)^{2}}{b}=1
 \end{cases}}
$$

Где, $a$ и $b$ показаны на рисунке \ref{elipsoid}.
\addimghere{elipsoid}{0.7}{Эллипс.}{elipsoid} 

Параметры $a$ и $b$, первый раз выбираются минимально возможными для возможности построения безопасного цилиндрической трубки, через точки на эллипсе, не пересекающейся с замеченным препятствием.
На рисунке \ref{elipsoidMat} красными точками обозначены возможные центры будущей траектории.
Всего точек 20 и они выбираются равномерно по всему эллипсу.
Обход начинается с крайней левой точки, по часовой стрелке.
Координаты этих точек вычисляются по следующей формуле:
$$
{\begin{cases}X_i=X_c-a\,\cos t\\Y_i=Y_c-b\,\sin t\\Z_i=Zc\end{cases}}\;\;\; t =\frac{2\pi i}{20} 
$$
\addimghere{elipsoidMat}{0.7}{Эллипс.}{elipsoidMat}
Если во время обхода всех точек не было найдено ни одной подходящей, то параметры эллипса увеличиваются на расстояние $\delta r$, вычисляемому по формуле:
$$
\delta r = \frac{R_{sv}}{4}
$$
Увеличение размеров эллипса происходит до тех пор, пока не будет найдена подходящая точка или размер эллипса не превысит широту обзора 3Д-камеры.
Если не нашлось точки через которую возможен облет препятствия, то БЛА переходит в режим ожидания оператора.
В задачах с ограниченным управлением, например невозможно пролетать под препятствиями, возможно ограничить поиск конкретной частью эллипса.

\subsection{Алгоритм RRT*}


\begin{enumerate}
	\item Выбирается начальная точка и она определяется как корень дерева;
	\item повторять пока в дереве нет узла содержащего конечную точку:
	\begin{enumerate}
		\item если существует линия соединяющая вновь добавленный узел и конечную точку не пересекающая препятствия, то добавить к дереву новый узел расположенный на этой линии и находящийся на расстоянии шага. Прервать текущую итерацию;
		\item создать случайно расположенный семя;
		\item для существующих узлов на дереве:
		\begin{enumerate}
			\item вычислить дистанцию между семенем и узлом;
			\item добавить это расстояние к общему, которое хранится в каждом узле;
			\item найти родительский узел с минимальным общим расстоянием;
		\end{enumerate}
		\item добавить семя в дерево как новый узел, если не выполняются все следующие условия:
		\begin{enumerate}
			\item семя за пределами рабочей зоны;
			\item на пути от выбранного узла к семени есть препятствия;
			\item расстояние между семенем и выбранным узлом больше максимального;
			\item линия соединяющая выбранный узел и семя пересекает другой узел;
		\end{enumerate}
		\item сохранить в новый узел:
		\begin{enumerate}
			\item расстояние между новым узлом и корнем дерева;
			\item дистанцию между новым узлом и корнем дерева;
			\item номер родительского узла;
		\end{enumerate}
	\item закончить повторение;
	\end{enumerate}
	
\end{enumerate}

\clearpage
